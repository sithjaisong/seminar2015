\documentclass[12pt,a4paper]{article}
\usepackage{graphicx}
\usepackage[applemac]{inputenc} %% european characters can be used (Mac OS)
% ------------------------------------------------
\usepackage[T1]{fontenc}   %% get hyphenation and accented letters right
\usepackage{mathptmx}      %% use fitting times fonts also in formulas
\pagestyle{empty}                %% no page numbers!
\usepackage[left=35mm, right=35mm, top=15mm, bottom=20mm, noheadfoot]{geometry}

% begin the document
\begin{document}
\thispagestyle{empty}

\title{\textbf{What data science can offer a botanic epidemiologist}}
\author{Sith Jaisong \\
Plant Disease Management Group, CESD, IRRI\\ Los Ba\~{n}os, Philippines\\
s.jaisong@irri.org}
\date{} % <--- leave date empty
\maketitle\thispagestyle{empty} %% <-- you need this for the first page
% introduction should have the objective or the problem or even telling the reader what you wanna say, and 

% I'd like you to follow the American Phytopathological Society guidelines for abstract submission

%Presentation Title
%Capitalize only the first letter of the first word and any proper nouns, (e.g., Effect of pesticides on recovery of Didymella bryoniae from cucurbit vines). The title is limited to 150 characters including spaces. (Approximately 30 word count.) Registered names and trademarks are not permitted in title.
%
%Sith: I totally forget that you used to mention  the abstract should be following the role from American Phytopathology socity
%
%Abstract Text
%Read the technical requirements and view the sample abstract before submitting your abstract.
%
%The abstract must be in one paragraph.
%DO NOT include the title, author name(s), or author affiliations in the abstract text field.
%Copy the abstract and paste it into the submission form abstract text box under the Abstract Copy/Body field header.
%Or type text in to the abstract field.
%Use the abstract toolbar to add formatting (italics, superscripts, subscripts, Greek or math symbols), or use the start coding <i> and the stop coding </i>, if you prefer.
%If the symbol is not available, spell it out (e.g., theta).
%Character limit is 1,490 characters including spaces (Approximately 250 word count).
%
%
%Sample Abstract
%Didymella bryoniae, the fungus that causes gummy stem blight, survives between crops in cucurbit debris. A pesticide that eliminates the fungus from infested debris would reduce initial inoculum for subsequent crops planted in infested fields. Naturally infected, 5-cm muskmelon vine sections were sprayed with field-equivalent rates of three herbicides, four fungicides, six salts, three botanical extracts, or three organic pesticides. After 3 days, vine sections were cut into 1-cm pieces and cultured on 1/4 PDA plus antibiotics. Each pesticide was tested 2 to 4 times with 10 to 20 vine sections per treatment. Chlorothalonil, mancozeb, sodium bisulfite, and pyraclostrobin + boscalid (Pristine) consistently reduced recovery of D. bryoniae to an average of 63, 57, 41, and 8% of vine pieces, respectively, compared to a water-treated control (99%). The other pesticides did not significantly reduce recovery of the fungus. Using Pristine to treat debris at the end of the season is not advisable, because of the risk of resistance to this fungicide. However, a non-specific material, such as a broad-spectrum fungicide or a salt, could be used to reduce the amount of surviving inoculum.


Whereas a traditional data analyst may look only at data from a single source (a set of experiments), data science enables us explore and examine data from multiple disparate sources. Data science is the science adopting techniques and theories from broad areas of mathematics, statistics and computer science for extracting information from large and complex data. Scientists are facing challenges handling and processing the data due to the amount and complexity of data. In order to manipulate these data, data science is very helpful and also it is equipped with variety of techniques from statistics.  As botanic epidemiologists, we explore the causes of plant disease epidemics. To investigate the causes, we analyze the data by combining weather or climatic data (e.g., temperature, relative humidity) and relevant variables that we think can be used to explain why disease occurs. This process creates more data, and consequently may lead to a new extensive result. Because either the data or the tools to analyze the data didn't exist before, data science potentially gives botanic epidemiologists the opportunities to discover previously hidden insights into why diseases occur. In this seminar, Sith will present how scientists apply the ideas and processes from data science, give some examples of the result applied data science. Finally, Sith will show the possible ways that data science can involve in botanical epidemiological studies.


\end{document}




















